\begin{frame}{tableau.length}
    Le champ $length$ permet de connaître le nombre d'éléments d'un tableau
    \lstinputlisting[firstline=3, lastline=4, firstnumber=3]{java/Length.java}
\end{frame}

\begin{frame}{A quoi ça sert?}
    Prenons comme exemple l'exercice 96 du syllabus.
    \begin{block}{Exercice}
        Écrire un algorithme qui déclare un tableau de 100 entiers
        et initialise chaque élément à la valeur de son indice.
        Ainsi, la case numéro $i$ contiendra la valeur $i$.
    \end{block}
    Tout d'abord, il faut déclarer et initialiser un tableau de 100 entiers.
    \lstinputlisting[firstline=3, lastline=3, firstnumber=3]{java/ParcourirTableau.java}
    Ensuite il faut parcourir le tableau et initialiser la valeur de chaque element à son indice.
    \lstinputlisting[firstline=5, lastline=7, firstnumber=5]{java/ParcourirTableau.java}
\end{frame}