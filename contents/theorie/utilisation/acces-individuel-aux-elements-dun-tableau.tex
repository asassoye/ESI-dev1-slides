\begin{frame}{Syntaxe}
    \pause
    On peut accéder à une valeur particulière en indiquant l'indice entre crochets.
    \lstinputlisting[firstline=3, lastline=4, firstnumber=3]{java/Acces.java}

    \pause
    \begin{alertblock}{ArrayIndexOutOfBoundsException}
        Si on essaye d'accéder à un indice négatif ou en dehors de la limite du tableau
        \lstinputlisting[firstline=6, lastline=6, firstnumber=6]{java/Acces.java}
        nous avons une exception lors de l'exécution:
        \lstinputlisting[language=bash]{java/error.txt}
    \end{alertblock}

\end{frame}

\stepcounter{exercice}
\begin{frame}{Exercice \theexercice}{Syllabus Exercice 95}
    Écrire un algorithme qui déclare un tableau de 100 chaines
    et met votre nom dans la 3\ieme{} case du tableau.
\end{frame}

\stepcounter{exercice}
\begin{frame}{Exercice \theexercice}{Syllabus Exercice 98}
    Expliquez la différence entre $tab[i] = tab[i+1]$
    et $tab[i] = tab[i]+1$.
\end{frame}

\begin{frame}{Exercice \theexercice~- Solution}{Syllabus Exercice 98}
    \pause
    \begin{block}{\textbf{$tab[i] = tab[i+1]$}}
        Ici, la valeur de $tab[i+1]$ est affectée à $tab[i]$
    \end{block}

    \pause
    \begin{block}{\textbf{$tab[i] = tab[i]+1$}}
        Ici, la valeur de $tab[i]$ est incrémentée puis est affectée à $tab[i]$
    \end{block}

    \pause
    \begin{exampleblock}{Equivalence}
        $tab[i] = tab[i]+1$ reviendrait au meme que de faire
        \begin{itemize}
            \item tab[i]++
        \end{itemize}
    \end{exampleblock}
\end{frame}