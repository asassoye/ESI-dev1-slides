\begin{frame}{Forme de base}
    La déclaration d'un tableau se fait en ajoutant des crochets à côté du type
    \lstinputlisting[firstline=3, lastline=3, firstnumber=3]{java/Declaration.java}

    \begin{exampleblock}{Types}
        En Java, les éléments d'un tableau peuvent être de n'importe quel type primitif.
    \end{exampleblock}

    \lstinputlisting[firstline=4, lastline=4, firstnumber=4]{java/Declaration.java}

\end{frame}

\begin{frame}{Variantes}
    On peut déclarer plusieurs tableaux sur une seule ligne
    \lstinputlisting[firstline=5, lastline=5, firstnumber=5]{java/Declaration.java}
    Nous pouvons aussi déclarer un tableau sur la même ligne que de simple variables (de même type).
    Dans ce cas, les crochets se mettent derrière le nom de la variable.
    \lstinputlisting[firstline=6, lastline=6, firstnumber=6]{java/Declaration.java}
\end{frame}

\begin{frame}{Elements de type 'objet'}
    Si l'on a défini une classe de type $Point$, nous pouvons également déclarer un tableau de points.

    \lstinputlisting[firstline=7, lastline=7, firstnumber=7, morekeywords=Point]{java/Declaration.java}
\end{frame}