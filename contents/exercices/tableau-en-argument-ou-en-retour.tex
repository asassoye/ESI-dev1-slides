\stepcounter{exercice}
\begin{frame}{Exercice \theexercice}{Syllabus Exercice 101}
    Écrire un algorithme qui reçoit en paramètre le tableau $integers$
    de $n$ entiers et qui retourne la somme de ses éléments.
\end{frame}

\stepcounter{exercice}
\begin{frame}{Exercice \theexercice}{Syllabus Exercice 102}
    Écrire un algorithme qui reçoit en paramètre le tableau $doubles$ de
    $n$ réels et qui retourne le nombre d’éléments du tableau.
\end{frame}

\stepcounter{exercice}
\begin{frame}{Exercice \theexercice}{Syllabus Exercice 104}
    \begin{enumerate}
        \item Écrire un algorithme qui reçoit en paramètre le tableau
        $cotes$ de $n$ entiers représentant les cotes des étudiants
        et qui retourne un booléen indiquant s’il contient \textbf{au
        moins} une fois la valeur 20.

        \item Écrire un algorithme qui reçoit en paramètre le tableau
        $cotes$ de $n$ entiers représentant les cotes des étudiants
        et qui retourne un booléen indiquant s’il contient
        \textbf{exactement} une fois la valeur 20.

    \end{enumerate}
\end{frame}

\stepcounter{exercice}
\begin{frame}{Exercice \theexercice}{Syllabus Exercice 105}
    Écrire un algorithme qui lance $n$ fois deux dés et compte le nombre de
    fois que chaque somme apparait. Cet algorithme retourne un tableau
    d'entiers\,; un élément par somme.
\end{frame}

