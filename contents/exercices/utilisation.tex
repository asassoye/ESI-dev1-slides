\stepcounter{exercice}
\begin{frame}{Exercice \theexercice}{Syllabus Exercice 97}
    Écrire un algorithme qui déclare un tableau de 100 entiers
    et y met les nombres de 1 à 100.
\end{frame}

\begin{frame}{Exercice \theexercice~- Solution}{Syllabus Exercice 97}
    \lstinputlisting[morekeywords=Point]{java/AlgorithmeInitialisation.java}
\end{frame}

\stepcounter{exercice}
\begin{frame}{Exercice \theexercice}{Syllabus Exercice 98}
    Expliquez la différence entre $tab[i] = tab[i+1]$
    et $tab[i] = tab[i]+1$.
\end{frame}

\stepcounter{exercice}
\begin{frame}{Exercice \theexercice}{Syllabus Exercice 99}
    Écrire les entêtes (et uniquement les entêtes)
    des algorithmes qui résolvent les problèmes suivants~:
    \begin{enumerate}
        \item
        Écrire un algorithme qui
        inverse le signe de tous les éléments négatifs dans un tableau d’entiers.
        \item
        Écrire un algorithme qui
        donne le nombre d’éléments négatifs dans un tableau d’entiers.
        \item
        Écrire un algorithme qui
        détermine si un tableau d’entiers contient au moins un nombre négatif.
        \item
        Écrire un algorithme qui
        détermine si un tableau de chaines contient
        une chaine donnée en paramètre.
        \item
        Écrire un algorithme qui
        détermine si un tableau de chaines contient
        au moins deux occurrences de la même chaine,
        quelle qu’elle soit.
        \item
        Écrire un algorithme qui
        retourne un tableau donnant les $n$ premiers nombres premiers,
        où $n$ est un paramètre de l’algorithme.
        \item
        Écrire un algorithme qui
        reçoit un tableau d’entiers
        et retourne un tableau de booléens de la même taille
        où la case $i$ indique si oui ou non
        le nombre reçu dans la case $i$ est strictement positif.
    \end{enumerate}
\end{frame}

\stepcounter{exercice}
\begin{frame}{Exercice \theexercice}{Syllabus Exercice 100}
    Écrire un algorithme qui
    inverse le signe de tous les éléments négatifs dans un tableau d’entiers.
\end{frame}

\stepcounter{exercice}
\begin{frame}{Exercice \theexercice}{Syllabus Exercice 103}
    Écrire un algorithme qui
    donne le nombre d’éléments négatifs dans un tableau d’entiers.
\end{frame}

\stepcounter{exercice}
\begin{frame}{Exercice \theexercice}{Syllabus Exercice 106}
    Écrire un algorithme qui reçoit un numéro de mois (de 1 à 12)
    ainsi qu’une année et donne le nombre de jours dans ce mois
    (en tenant compte des années bissextiles).
    N’hésitez pas à réutiliser des algorithmes déjà écrits.
\end{frame}
