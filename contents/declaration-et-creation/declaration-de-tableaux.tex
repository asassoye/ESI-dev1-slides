\begin{frame}{Forme de base}
    La déclaration d'un tableau se fait comme la déclaration d'une variable suivi de crochets
    \lstinputlisting[firstline=3, lastline=3, firstnumber=3]{contents/declaration-et-creation/java/Declaration.java}

    \begin{exampleblock}{Types}
        En Java, les éléments d'un tableau peuvent être de n'importe quel type primitif.
    \end{exampleblock}

    \lstinputlisting[firstline=4, lastline=4, firstnumber=4]{contents/declaration-et-creation/java/Declaration.java}

\end{frame}

\begin{frame}{Variantes}
    On peut également mettre les crochets avant le nom
    \lstinputlisting[firstline=5, lastline=5, firstnumber=5]{contents/declaration-et-creation/java/Declaration.java}
    Ceci permet de pouvoir déclarer plusieurs tableaux sur une ligne
    \lstinputlisting[firstline=6, lastline=6, firstnumber=6]{contents/declaration-et-creation/java/Declaration.java}
\end{frame}

\begin{frame}{Elements de type 'objet'}
    Si l'on a défini une classe de type $Point$, nous pouvons également déclarer un tableau de points.

    \lstinputlisting[firstline=7, lastline=7, firstnumber=7, morekeywords=Point]{contents/declaration-et-creation/java/Declaration.java}
\end{frame}