\documentclass{beamer}

\usepackage{./styles/esi-beamer}


\title{Chapitre 10: Les tableaux}
\subtitle{\tiny [DEV1] Cours de développement}
\author{Andrew SASSOYE}
\date{2019-2020}


\begin{document}
	\begin{frame}
		\titlepage
	\end{frame}

	\begin{frame}
		\begin{center}
			\includegraphics[width=8mm]{./styles/images/cc}
			\includegraphics[width=8mm]{./styles/images/by}
			\includegraphics[width=8mm]{./styles/images/nc}
			\includegraphics[width=8mm]{./styles/images/sa}
		\end{center}

		\begin{block}{Licence}
			\small Cette \oe uvre est mise à disposition sous licence Creative Common
			Attribution - Pas d\rq Utilisation Commerciale - Partage dans les Mêmes Conditions 4.0 International.
		\end{block}

		\tiny Pour voir une copie de cette licence, visitez
		\href{https://creativecommons.org/licenses/by-nc-sa/4.0/}{https://creativecommons.org/licenses/by-nc-sa/4.0/}
	\end{frame}

    \part{Théorie}
    \frame{\partpage}

    \section{Déclaration et création}\label{sec:declaration-et-creation}
        \subsection{Introduction}\label{subsec:introduction}
            \begin{frame}
    \begin{block}{Definition}
        En programmation,
        on parle de tableau pour d\'esigner un ensemble d'\'el\'ements de m\^eme type design\'e par un nom unique,
        chaque \'el\'ement \'etant rep\'er\'e par un indice pr\'ecisant sa position au sein de l'ensemble
    \end{block}
\end{frame}
        \subsection{Déclaration de tableaux}\label{subsec:declaration-de-tableaux}
            \begin{frame}[containsverbatim]
    \begin{lstlisting}[language=Java, frame=single]
    int t[];
    \end{lstlisting}
\end{frame}
        \subsection{Création de tableaux}\label{subsec:creation-de-tableaux}
            \begin{frame}{Cr\'eation par l'operateur new}

\end{frame}

\begin{frame}{Utilisation d'un initialiseur}

\end{frame}

    \section{Utilisation}
        \subsection{Accès individuel aux éléments d'un tableau}\label{subsec:acces-individuel-aux-elements-dun-tableau}
        \subsection{Affectation de tableaux}\label{subsec:affectation-de-tableaux}
        \subsection{La taille d'un tableau: length}\label{subsec:la-taille-dun-tableau}
        \subsection{Boucle for\ldots each}\label{subsec:boucle-foreach}

    \section{Tableau en argument ou en retour}\label{sec:tableau-en-argument-ou-en-retour}

    \section{Tableaux à plusieurs indices}\label{sec:tableaux-a-plusieurs-indices}
        \subsection{Présentation générale}\label{subsec:presentation-generale}
        \subsection{Initialisation}\label{subsec:initialisation}
        \subsection{Exemple}\label{subsec:exemple}
        \subsection{For\ldots each}\label{subsec:for-each}

    \part{Exercices}
    \frame{\partpage}

\end{document}